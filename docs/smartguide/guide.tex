\documentclass[12pt,a5paper]{book}
\usepackage{listings}
\usepackage{xcolor}
\usepackage{hyperref}

\lstset{
    backgroundcolor=\color{lightgray},
    basicstyle=\ttfamily,
    breaklines=true
}

\lstdefinelanguage{JavaScript}{
  morekeywords=[1]{break, continue, delete, else, for, function, if, in,
    new, return, this, typeof, var, void, while, with},
  % Literals, primitive types, and reference types.
  morekeywords=[2]{false, null, true, boolean, number, undefined,
    Array, Boolean, Date, Math, Number, String, Object},
  % Built-ins.
  morekeywords=[3]{eval, parseInt, parseFloat, escape, unescape},
  sensitive,
  morecomment=[s]{/*}{*/},
  morecomment=[l]//,
  morecomment=[s]{/**}{*/}, % JavaDoc style comments
  morestring=[b]',
  morestring=[b]"
}[keywords, comments, strings]

\title{A Smart Guide to Building Web Applications}
\author{João Costa Seco}
\begin{document}

\maketitle

\chapter{Introduction}

This small booklet has X parts illustrating small aspects of the construction of
a web server in JavaScript, a static web page using basic technologies like HTML
and CSS. And then how to develop a dynamic web page with events and dynamic
construction of HTML elements. 

The goal of this booklet is to illustrate the different aspects of the
construction of a web server in TypeScript. The booklet is not intended to be a
complete reference, but rather a quick guide to get you started.

\part{A Web Server in TypeScript}

\chapter{Setting up your first node application}

A Web server is a program that listens to requests from a browser and returns a
response. Responses can be HTML pages, images, JSON data, etc. Content can be
either static if provided by an existing file in the server filesystem, or
dynamic if it is created on demand and based on the information from the request.

We are going to use Node.js to create our web server. Node.js is a JavaScript
runtime environment that allows us to run JavaScript and TypeScript code outside
of the browser.

\begin{enumerate}

\item 
To start with you need to install Node.js. You can download it from the
\href{https://nodejs.org}{Node.js} website.

\item Next, you need to create a directory for your project and initialize it.

\begin{lstlisting}[language=bash]
mkdir myproject
cd myproject
npm init -y
\end{lstlisting}

These commands will add a \texttt{package.json} file to your project. This file
contains all the configuration information for your project.

\item Next, you need to install the TypeScript compiler and initialize it.

\begin{lstlisting}[language=bash]
npm install -g typescript
tsc --init
npm install ts-node typescript --save-dev
\end{lstlisting}

These commands will add the file \texttt{tsconfig.json} to your project and
install the necessary packages. This will contain the information for the
TypeScript compiler to compile your code.

\item Next, you need to write your first TypeScript file. Create a file called
\texttt{index.ts} in a folder called \texttt{src}.

\begin{lstlisting}
console.log("Hello World");
\end{lstlisting}

\item Change your \texttt{package.json} file and modify the "start" script to run your program.

\begin{lstlisting}[language=JavaScript]
"scripts": {
    "start": "ts-node src/index.ts"
},
\end{lstlisting}

\item Finally, you can run your program with the following command:

\begin{lstlisting}[language=bash]
npm start
\end{lstlisting}

You should see the message "Hello World" printed in the console.
Congratulations, you have succeded in writing and running your first TypeScript
application.

\end{enumerate}

\chapter{Setting up your first web server}

\begin{enumerate}
\item Next, you need to install the Express framework

\begin{lstlisting}[language=bash]
    npm install express --save
\end{lstlisting}

\end{enumerate}

\chapter{Loading files from the web server}

\chapter{Returning dynamic content}

Use a form to upload parameters.

\chapter{Webservices: returning JSON data}

\part{A Web page using HTML and CSS}

\chapter{HTML: Writing your first web page}

\chapter{HTML: The anatomy of an HTML element}

\chapter{HTML: Basic Text Formatting}

\chapter{HTML: Adding images}

\chapter{HTML: Linking different pages}

\chapter{HTML: Formatting with stylesheets}

\chapter{HTML: Basic Layout with CSS}

\chapter{HTML: Making requests with Forms}

\part{Reactive pages: Events and JavaScript}

\part{Dynamic pages with JavaScript}

\end{document}